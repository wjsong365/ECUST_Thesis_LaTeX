%!TEX program = xelatex


% 打印选项: 双面打印 twoside;单面打印 oneside
% 模板选项: 硕士论文 master; 博士论文 doctor
% openright, openany 指定新的一章 \chapter 是在奇数页(右侧)开始,还是直接紧跟着上一页开始。report 默认为 openany,book 默认为 openright。对 article 无效。
\documentclass[twoside, doctor, openany]{ecust_thesis}
\usepackage{amssymb,amsfonts,amsmath,mathrsfs}
\allowdisplaybreaks[4] % 长公式换页
\usepackage{bm} % 加粗数学公式

\AtBeginDocument{% 修改公式上下间距 https://tex.stackexchange.com/questions/69662/how-to-globally-change-the-spacing-around-equations
 \abovedisplayskip=5pt plus 4pt minus 2pt
 \abovedisplayshortskip=5pt plus 4pt minus 4pt
 \belowdisplayskip=5pt plus 4pt minus 2pt
 \belowdisplayshortskip=5pt plus 4pt minus 4pt
}

\usepackage{enumerate}
\usepackage{multirow}
\usepackage{algpseudocode} % 算法包
\usepackage{algorithmicx} % 算法包
\usepackage[chapter]{algorithm} % 算法按章编号
% \renewcommand{\thealgorithm}{\arabic{chapter}.\arabic{algorithm}}
%\usepackage{arydshln} % 在引用longtable包的情况下,arydshln包会和booktab包冲突
\usepackage[list=off]{bicaption}% ccaption -- bicaption
\usepackage{longtable}
\setlength{\LTpre}{0pt} % 设置longtable的前后间距 https://tex.stackexchange.com/questions/5683/how-to-remove-top-and-bottom-whitespace-of-longtable
\setlength{\LTpost}{0pt}
\usepackage{threeparttable}
\usepackage{tabularx}
\makeatletter
\newcommand{\multiline}[1]{% 表格中的多行
  \begin{tabularx}{\dimexpr\linewidth-\ALG@thistlm}[t]{@{}X@{}}
    #1
  \end{tabularx}
}
\newcommand{\tabincell}[2]{\begin{tabular}{@{}#1@{}}#2\end{tabular}}
\makeatother
\usepackage[figuresright]{rotating} % 用于旋转表:
\usepackage{float} % 设置表格位置 \begin{table}[H] % 用H表示必须放在这里
% \restylefloat{table} % https://stackoverflow.com/questions/1673942/latex-table-positioning
\usepackage{balance}
\usepackage{autobreak}
\usepackage{graphicx}
\usepackage{epsfig}
\usepackage{color}
\usepackage[normalem]{ulem} % underline
\usepackage{CJKulem} % 使用Ctex,ulem宏包中下划线命令\uline如果对中文处理,则中文换行失效,需要换成一下Ctex专用宏包。
\usepackage[utf8x]{inputenc}
\usepackage{textcomp}

\usepackage{pdfpages} % 用于合并pdf

% \usepackage[bookmarksopen=true]{hyperref}
\usepackage{bookmark} % 把不编号章节仅加入PDF书签,不放入目录

\usepackage{paralist}
\let\itemize\compactitem
\let\enditemize\endcompactitem
\let\enumerate\compactenum
\let\endenumerate\endcompactenum
\let\description\compactdesc
\let\enddescription\endcompactdesc
\setdefaultleftmargin{2em}{2em}{1em}{1em}{1em}{1em} % 调整itemize对齐

\usepackage{titlesec} % 设置垂直对齐方式,\flushbottom垂直两端对齐(标题不会出现在页尾),\raggedbottom顶部对齐
\raggedbottom

% \renewcommand{\thefootnote}{\fnsymbol{footnote}} % footnote格式https://www.overleaf.com/learn/latex/Footnotes

%% 定义常用的符号
\newcommand{\ssd}{{$^\circ$C}} %℃
\newcommand{\hsd}{{$^\circ$F}} %℉
\newcommand{\blh}{\textasciitilde} % 波浪号~

% \theoremstyle{plain} % 默认,这几个命令改变remark中的字体样式
% \theoremstyle{remark}
\theoremstyle{definition}
\newtheorem{definition}{定义}[chapter]
\newtheorem{remark}{注}[chapter]

%\allowdisplaybreaks % 使大公式跨页显示,以补充前一页底部的空白。
\interdisplaylinepenalty=2500
\hyphenpenalty=5000 % 通过调整单词断开位置调整段落右边对齐效果
\tolerance=1200

\renewcommand{\algorithmicrequire}{\textbf{Input:}}
\renewcommand{\algorithmicensure}{\textbf{Output:}}
%\renewcommand{\algorithmicfor}{\textbf{对}}
%\renewcommand{\algorithmicif}{\textbf{如果}}
%\renewcommand{\algorithmicdo}{\textbf{执行}}
%\renewcommand{\algorithmicelse}{\textbf{否则}}
%\renewcommand{\algorithmicthen}{\textbf{则}}
%\renewcommand{\algorithmicend}{\textbf{结束}}

\begin{document}

%%%%%%%%%%%%%%%%%%%%%%%%%%%%%%
%% 封面
%%%%%%%%%%%%%%%%%%%%%%%%%%%%%%

\title{华东理工大学博士论文LaTeX模板}
\englishtitle{
  \begin{tabular}{c}
    两行英文标题 \\ 换行方式
  \end{tabular}}
\englishauthor{}

% \includepdf[pages=-]{chapters/硕、博士封面.pdf}
% \includepdf[pages=-]{chapters/空白页.pdf}
% \includepdf[pages=-]{chapters/学位论文扉页.pdf}
% \includepdf[pages=-]{chapters/学位论文授权声明.pdf}
% \includepdf[pages=-]{chapters/空白页.pdf}
% \includepdf[pages=-]{chapters/学位论文提交要求.pdf}
% \includepdf[pages=-]{chapters/空白页.pdf}
% \includepdf[pages=-]{chapters/作者声明.pdf}
% \includepdf[pages=-]{chapters/空白页.pdf}
%% 上面资料下载地址:https://gschool.ecust.edu.cn/12735/list.htm

%%%%%%%%%%%%%%%%%%%%%%%%%%%%%%
%% 前置部分
%%%%%%%%%%%%%%%%%%%%%%%%%%%%%%
\frontmatter

% 摘要

\begin{abstract}
本模板针对有一定LaTeX基础的同学,提供了华东理工大学博士学位论文写作模板,简要给出了图、表、算法等示例。
如果读者对LaTeX还不熟悉,建议以参考文献\cite{LaTeX2e介绍}作为入门。


\keywords{华东理工大学;毕业论文;LaTeX}
\end{abstract}



\begin{englishabstract}
This is English abstract.

\englishkeywords{AAA; BBB; CCC}

\end{englishabstract}



% 加入目录
\tableofcontents
\cleardoublepage % 保证第一章的位置
%% 加入表格索引
%%\listoftables
% 加入插图索引
%\listoffigures


%%%%%%%%%%%%%%%%%%%%%%%%%%%%%%
%% 正主体部分
%%%%%%%%%%%%%%%%%%%%%%%%%%%%%%
\mainmatter

%% 各章正文内容

\chapter{模板说明}\label{c1:intro}

本模板基于\verb|北京理工大学硕士(博士)学位论文LaTeX模板|\upcite{BITthesis}修改。
修改内容主要包括:
\begin{enumerate}
  \item 页眉页脚样式、高度、页边距;
  \item 左右页边距;
  \item 行间距;
  \item 公式、图、表前后间距;
  \item 各级标题前后间距和样式;
  \item 列表、枚举的缩进;
  \item 其他样式等。
\end{enumerate}

下面给出论文中常用的各类图、表、算法、公式、参考文献等的使用方法。

\begin{remark}
  本文默认读者已经知道了LaTeX的基本语法,并可以熟练使用LaTeX写期刊论文。
  下面仅介绍重要的、常用的示例,并不加以详细解释。
\end{remark}

\section{编译}

编译顺序为:xelatex-->bibtex-->xelatex-->xelatex。

推荐使用VSCode + LaTeX Workshop的组合写论文。

\section{图}\label{c1sec:figure}

图\ref{c1fig_demo1}是一个标准的图。
\begin{figure}[!htb]
  \centering
  \includegraphics[width=5cm]{figures/chapter1/fig_demo.pdf}
  \bicaption{示意图}
  {Demo}
  \label{c1fig_demo1}
\end{figure}

图\ref{c1fig_demo2}是一行多列的子图安排方式。
\begin{figure}[!htb]
  \centering
  \subfigure[]{\includegraphics[width=0.28\textwidth]{figures/chapter1/fig_demo.pdf}}
  \subfigure[]{\includegraphics[width=0.28\textwidth]{figures/chapter1/fig_demo.pdf}}
  \bicaption{一行多列的子图安排方式。
  (a) 子图1,
  (b) 子图2}
  {Subfigure. 
  (a) subfigure 1,
  (b) subfigure 2} 
  \label{c1fig_demo2}
\end{figure}

图\ref{c1fig_demo3}是多行多列的子图安排方式。如果想让这段话出现在图\ref{c1fig_demo2}之后,则可以在图\ref{c1fig_demo2}中使用\verb|\begin{figure}[H]|命令。
\begin{figure}[H]
  \centering
  \subfigure[当然,如果你想,这里也可以写些内容]{\includegraphics[width=0.28\textwidth]{figures/chapter1/fig_demo.pdf}}
  \subfigure[]{\includegraphics[width=0.28\textwidth]{figures/chapter1/fig_demo.pdf}}\\
  \subfigure[]{\includegraphics[width=0.28\textwidth]{figures/chapter1/fig_demo.pdf}}
  \subfigure[]{\includegraphics[width=0.28\textwidth]{figures/chapter1/fig_demo.pdf}}
  \bicaption{子图示意图。
  (a) 子图1,
  (b) 子图2,
  (c) 子图3,
  (d) 子图4}
  {Subfigure. 
  (a) subfigure 1,
  (b) subfigure 2,
  (c) subfigure 3,
  (d) subfigure 4} 
  \label{c1fig_demo3}
\end{figure}

% \newpage
\section{表格}\label{c1sec:table}

表\ref{c1tab_1}是一个标准的三线表格。
\begin{table}[H]
  \zihao{5} 
  \renewcommand{\arraystretch}{1.3}
  \renewcommand{\tabcolsep}{5pt}
  \bicaption{这是一个标准表格}
  {This is a standard table}
  \label{c1tab_1}
  \centering
  \begin{tabular}{cc}
  \toprule
  名称1 & 名称2 \\
  \midrule
  A	&	B	\\
  A	&	B	\\
  \bottomrule
  \end{tabular}
\end{table}

表\ref{c1tab_2}是一个带脚注的复杂表格。可以看到这段文字没有出现在上一页的末尾(尽管上一页的末尾还有空间),是因为表\ref{c1tab_1}中使用了\verb|\begin{table}[H]|命令。
\begin{table}[!htb]
  \zihao{5} 
  \renewcommand{\arraystretch}{1.3}
  \renewcommand{\tabcolsep}{5pt}
  \bicaption{这是一个带脚注的复杂表格\textsuperscript{a}}
  {This is a complex table}
  \label{c1tab_2}
  \centering
  \begin{tabular}{ccccccc}
  \toprule
  \multirow{3}{*}{AAA}	&	\multicolumn{2}{c}{BBB\textsuperscript{b}}	&	\multicolumn{2}{c}{CCC}	&	\multicolumn{2}{c}{DDD}	\\
  \cmidrule(r){2-3} \cmidrule(r){4-5} \cmidrule(r){6-7}
    &	\tabincell{c}{A1\\A2}	&	\tabincell{c}{B1\\B2}	&	\tabincell{c}{C1\\C2}	&	\tabincell{c}{D1\\D2}	&	\tabincell{c}{E1\\E2}	&	\tabincell{c}{F1\\F2}	\\
  \midrule
  AAA1	&	20361	&	27.02	&	20901	&	27.72	&	21711	&	28.14	\\
  AAA2	&	10051	&	13.34	&	10051	&	13.33	&	10311	&	13.36	\\
  \bottomrule
  \end{tabular}\\
  \textsuperscript{a} 脚注1 \\
  \textsuperscript{b} 脚注2 \\
\end{table}

表\ref{c1tab_3}是一个较宽的表格,需要缩放到页宽范围,这可以通过命令\\
\verb|\resizebox{1\textwidth}{!}{}|\\
实现。
此表格还采用了\verb|\specialrule{0pt}{3pt}{3pt}|命令来调整倒数第3行-倒数第2行的间距,使表格更美观。

\begin{table}[H]
  \zihao{5} 
  \bicaption{这是一个较宽的表格,需要缩放到页宽范围}
  {This is a complex table}
  \label{c1tab_3}
  \centering
  \resizebox{1\textwidth}{!}{ % 通过缩放,让宽表和正文一样宽
  \begin{tabular}{ccccccc}
  \toprule
  AAA	&	$\delta$	&	$\beta$\textsuperscript{\emph{a}}	&	\tabincell{c}{XXXXXXXXXXXX\\YYY}	&	\tabincell{c}{XXXXXXXXXXXX\\YYY}	&	\tabincell{c}{XXXXXXXXXXXX\\YYY}	&	\tabincell{c}{XXXXXXXXXXXX\\YYY}	\\
  \midrule
  \multirow{4}{*}{AAA}	&	2.549	&	2.549	&	30/30	&	30/30	&	30/30	&	30/30	\\
    &	AAA	&	2.718	&	30/30	&	30/30	&	30/30	&	30/30	\\
    &	AAA	&	3.094	&	30/30	&	30/30	&	30/30	&	30/30	\\
    &	AAA	&	3.200	&	30/30	&	30/30	&	30/30	&	30/30	\\ \specialrule{0pt}{3pt}{3pt}
  \multirow{2}{*}{BBB}	&	2.128	&	2.128	&	30/30	&	30/30	&	30/30	&	30/30	\\
    &	BBB	&	2.500	&	30/30	&	30/30	&	30/30	&	30/30	\\ 
  \bottomrule
  \end{tabular}}\\
  \textsuperscript{\emph{a}} 脚注1
\end{table}

当然,也可以将表\ref{c1tab_3}缩放到指定宽度,如80\%页面宽度,见表\ref{c1tab_4}。

\begin{table}[H]
  \zihao{5} 
  \bicaption{这是一个较宽的表格,需要缩放到80\%页宽范围}
  {This is a complex table}
  \label{c1tab_4}
  \centering
  \resizebox{0.8\textwidth}{!}{ % 通过缩放,让宽表和正文一样宽
  \begin{tabular}{ccccccc}
  \toprule
  AAA	&	$\delta$	&	$\beta$\textsuperscript{\emph{a}}	&	\tabincell{c}{XXXXXXXXXXXX\\YYY}	&	\tabincell{c}{XXXXXXXXXXXX\\YYY}	&	\tabincell{c}{XXXXXXXXXXXX\\YYY}	&	\tabincell{c}{XXXXXXXXXXXX\\YYY}	\\
  \midrule
  \multirow{4}{*}{AAA}	&	2.549	&	2.549	&	30/30	&	30/30	&	30/30	&	30/30	\\
    &	AAA	&	2.718	&	30/30	&	30/30	&	30/30	&	30/30	\\
    &	AAA	&	3.094	&	30/30	&	30/30	&	30/30	&	30/30	\\
    &	AAA	&	3.200	&	30/30	&	30/30	&	30/30	&	30/30	\\ \specialrule{0pt}{3pt}{3pt}
  \multirow{2}{*}{BBB}	&	2.128	&	2.128	&	30/30	&	30/30	&	30/30	&	30/30	\\
    &	BBB	&	2.500	&	30/30	&	30/30	&	30/30	&	30/30	\\ 
  \bottomrule
  \end{tabular}}\\
  \textsuperscript{\emph{a}} 脚注1
\end{table}


如果表格缩放到页面宽度后字体太小看不清,这时你可能需要通过使用\\
\verb|\begin{sidewaystable}|命令来实现一个横着放置的宽表,如表\ref{c1tab_5}所示。

\begin{sidewaystable}
  \zihao{5} 
  \renewcommand{\arraystretch}{1.2}
  % \renewcommand{\tabcolsep}{3pt}
  \bicaption{sidewaystable表格}
  {sidewaystable}
  \label{c1tab_5}
  \centering
  \begin{tabular}{ccc|cc|cc}
  \toprule
  Problem	&	AAA1	&	AAA2	&	BBB1	&	BBB2	&	CCC1	&	CCC2	\\
  \midrule
  ABC1	&	5.51e-01(8.20e-02)	&	\textbf{8.23e-01(1.75e-02)+ }	&	3.17e-02(3.99e-02)	&	\textbf{8.46e-01(9.72e-03)+ }	&	8.32e-01(9.47e-03)	&	\textbf{8.71e-01(1.51e-04)+ }	\\
  ABC2	&	2.38e-01(7.85e-02)	&	\textbf{4.81e-01(2.22e-02)+ }	&	0.00e+00(0.00e+00)	&	\textbf{9.77e-02(1.99e-01)+ }	&	4.48e-01(7.58e-02)	&	\textbf{5.38e-01(1.63e-04)+ }	\\
  ABC3	&	4.83e-01(6.85e-02)	&	\textbf{8.83e-01(9.03e-02)+ }	&	9.19e-02(6.93e-02)	&	\textbf{9.82e-01(3.03e-02)+ }	&	9.71e-01(1.49e-02)	&	\textbf{1.02e+00(3.01e-04)+ }	\\
  ABC4	&	0.00e+00(0.00e+00)	&	\textbf{5.83e-01(1.33e-01)+ }	&	0.00e+00(0.00e+00)	&	\textbf{9.58e-02(2.51e-01)+ }	&	1.55e-01(1.53e-01)	&	\textbf{6.77e-01(1.16e-01)+ }	\\
  ABC6	&	3.73e-01(7.27e-02)	&	\textbf{4.10e-01(2.62e-02)= }	&	1.39e-01(1.63e-01)	&	\textbf{1.79e-01(1.76e-01)= }	&	1.20e-01(7.40e-02)	&	\textbf{4.33e-01(1.39e-04)+ }	\\
  ABD1	&	1.21e+01(1.64e+00)	&	\textbf{4.31e+01(2.54e+00)+ }	&	1.58e+01(8.86e-01)	&	\textbf{4.49e+01(2.04e+00)+ }	&	2.74e+01(2.15e+00)	&	\textbf{3.34e+01(3.46e+00)+ }	\\
  ABD2	&	\textbf{5.54e+01(7.97e-01)}	&	5.51e+01(1.11e+00)= 	&	5.66e+01(5.68e-01)	&	\textbf{5.87e+01(3.75e-01)+ }	&	5.81e+01(4.11e-01)	&	\textbf{5.86e+01(3.19e-01)+ }	\\
  ABD4	&	2.86e+01(6.99e-01)	&	\textbf{3.27e+01(3.69e-01)+ }	&	3.06e+01(4.13e-01)	&	\textbf{3.40e+01(2.81e-01)+ }	&	3.38e+01(2.36e-01)	&	\textbf{3.49e+01(1.85e-01)+ }	\\
  ABD5	&	2.95e+01(3.10e-01)	&	\textbf{2.98e+01(4.45e-01)+ }	&	3.06e+01(2.38e-01)	&	\textbf{3.09e+01(2.48e-01)+ }	&	3.23e+01(2.38e-01)	&	\textbf{3.29e+01(1.74e-01)+ }	\\
  ABD6	&	2.47e+01(1.83e+00)	&	\textbf{2.83e+01(1.82e+00)+ }	&	2.67e+01(1.28e+00)	&	\textbf{2.98e+01(1.03e+00)+ }	&	3.08e+01(8.79e-01)	&	\textbf{3.12e+01(5.63e-01)= }	\\
  ABD7	&	2.87e+01(1.44e+00)	&	\textbf{3.25e+01(3.74e-01)+ }	&	3.22e+01(5.55e-01)	&	\textbf{3.41e+01(2.05e-01)+ }	&	3.40e+01(2.16e-01)	&	\textbf{3.44e+01(2.60e-01)+ }	\\
  ABD8	&	1.93e+01(2.52e+00)	&	\textbf{2.60e+01(8.95e-01)+ }	&	2.50e+01(6.55e-01)	&	\textbf{2.76e+01(2.78e-01)+ }	&	2.82e+01(3.08e-01)	&	\textbf{2.83e+01(3.76e-01)= }	\\
  ABD9	&	2.85e+01(1.72e+00)	&	\textbf{2.97e+01(1.87e+00)+ }	&	3.07e+01(1.21e+00)	&	\textbf{3.14e+01(1.74e+00)+ }	&	3.19e+01(7.18e-01)	&	\textbf{3.26e+01(3.73e-01)+ }	\\
  ABE1	&	4.93e-03(1.36e-02)	&	\textbf{1.09e-01(3.94e-02)+ }	&	2.70e-02(4.19e-02)	&	\textbf{1.09e-01(4.31e-02)+ }	&	5.89e-02(5.15e-02)	&	\textbf{8.66e-02(5.55e-02)+ }	\\
  ABE2	&	6.23e-01(2.53e-02)	&	\textbf{7.06e-01(1.98e-03)+ }	&	7.14e-01(2.33e-03)	&	\textbf{7.21e-01(2.34e-03)+ }	&	7.39e-01(8.54e-04)	&	\textbf{7.42e-01(4.94e-04)+ }	\\
  ABE3	&	0.00e+00(0.00e+00)	&	0.00e+00(0.00e+00)= 	&	\textbf{1.02e-03(5.60e-03)}	&	0.00e+00(0.00e+00)= 	&	0.00e+00(0.00e+00)	&	\textbf{1.18e-02(6.46e-02)= }	\\
  ABE4	&	5.26e-01(1.33e-01)	&	\textbf{5.34e-01(1.69e-01)= }	&	\textbf{4.76e-01(1.95e-01)}	&	4.57e-01(2.20e-01)= 	&	6.23e-01(1.66e-01)	&	\textbf{6.34e-01(1.42e-01)= }	\\
  ABE5	&	1.28e-01(6.25e-04)	&	\textbf{1.30e-01(1.27e-04)+ }	&	\textbf{1.26e-01(7.35e-04)}	&	1.25e-01(1.26e-03)- 	&	\textbf{1.29e-01(6.99e-04)}	&	1.28e-01(1.81e-03)= 	\\
  ABE6	&	1.09e-01(4.66e-02)	&	\textbf{1.29e-01(1.86e-03)+ }	&	1.17e-01(2.19e-02)	&	\textbf{1.18e-01(1.11e-02)+ }	&	\textbf{1.27e-01(1.04e-03)}	&	1.23e-01(1.14e-03)- 	\\
  ABE7	&	3.94e-01(2.29e-01)	&	\textbf{1.38e+00(3.38e-02)+ }	&	9.96e-02(1.76e-01)	&	\textbf{1.57e+00(3.60e-02)+ }	&	1.43e+00(5.21e-02)	&	\textbf{1.57e+00(1.15e-02)+ }	\\
  \midrule
  +/-/=	&		&	16/0/4	&		&	16/1/3	&		&	14/1/5	\\
  \bottomrule
  \end{tabular}
\end{sidewaystable}

\clearpage
如果一个表太长,则可能需要跨页表格,如表\ref{c1tab_long}所示。
\begin{center}
  \zihao{5} % 图、表、算法
  %\renewcommand{\arraystretch}{1.3} % longtable中的间距默认和正文一样
  \begin{longtable}{llrrr}
  \bicaption{这是一个长表}{This is a very long table}
  \label{c1tab_long} \\
  \toprule
  Name1	&	Name2	&	Name3	&	Name4	&	Name5	\\
  \midrule
  \endfirsthead
  \multicolumn{5}{r}{(接上表)}   \\
  \toprule
  Name1	&	Name2	&	Name3	&	Name4	&	Name5	\\
  \midrule
  \endhead
  AAA	&	A1	&	180.0	&	281.0	&	235.3	\\
      &	A2	&	5.3	&	43.5	&	35.1	\\
      &	A3	&	680.3	&	1020.0	&	759.3	\\
      &	A4	&	14.4	&	15.8	&	15.4	\\
      &	A5 	&	347.7	&	365.0	&	355.9	\\
      &	A6 	&	363.4	&	384.1	&	372.6	\\
      &	A7 	&	373.5	&	393.7	&	384.2	\\
      &	A8 	&	384.9	&	400.0	&	394.4	\\
      &	A9 	&	379.9	&	400.0	&	390.2	\\
      &	A10	&	379.5	&	399.7	&	390.2	\\
      &	A11	&	382.1	&	399.9	&	393.2	\\
  BBB	&	B1	&	895.8	&	918.8	&	907.2	\\
      &	B2	&	0.5	&	1.1	&	0.9	\\
      &	B3	&	715.6	&	2655.0	&	1360.1	\\
      &	B4	&	318.2	&	424.8	&	378.2	\\
      &	B5	&	183.0	&	242.0	&	215.0	\\
      &	B6	&	291.0	&	376.0	&	331.8	\\
      &	B7	&	373.0	&	452.0	&	429.6	\\
      &	B8	&	469.0	&	520.0	&	501.1	\\
      &	B9	&	521.0	&	565.0	&	549.5	\\
  CCC	&	C1&	912.7	&	947.5	&	929.2	\\
      &	C2	&	0.2	&	0.8	&	0.3	\\
      &	C3	&	143.0	&	229.0	&	184.7	\\
      &	C4	&	202.0	&	263.0	&	221.6	\\
      &	C5	&	243.0	&	304.0	&	269.9	\\
      &	C6	&	281.0	&	365.0	&	329.9	\\
      &	C7	&	293.0	&	379.0	&	342.3	\\
  DDD	&	D1	&	369.0	&	397.7	&	382.9	\\
      &	D2	&	384.6	&	406.7	&	394.6	\\
      &	D3	&	385.1	&	407.8	&	397.6	\\
      &	D4	&	388.3	&	405.1	&	398.7	\\
      &	D5	&	385.6	&	407.9	&	397.2	\\
      &	D6	&	387.4	&	414.5	&	401.6	\\
      &	D7	&	388.9	&	418.1	&	403.3	\\
      &	D8 	&	3.6	&	7.2	&	5.3	\\
      &	D9 	&	2.3	&	5.2	&	3.6	\\
      &	D10	&	11.8	&	23.9	&	17.6	\\
      &	D11	&	9.7	&	21.3	&	14.6	\\
      &	D12	&	34.4	&	42.6	&	39.2	\\
      &	D13	&	7.8	&	32.5	&	19.6	\\
  \bottomrule
  \end{longtable}
\end{center}%



\section{算法}\label{c1sec:algorithm}

下面是一个算法示例。

\begin{algorithm*}[!ht]
  \zihao{5} % 图、表、算法
\bicaption{算法示例}
{Algorithm demo}	
\label{c1alg_demo}
\begin{algorithmic}[1]
\Require{
$p_1$:参数1
\Statex	$p_1$:参数2
\Statex	$p_1$:参数3}
\Ensure{
$o_1$:输出1
\Statex	$o_2$:输出2}
\State    第一条命令   \Comment{第一条命令的注释}        
\While	{$True$}	
  \State /***** 下面代码块的注释 *****/
  \State 相关命令 \Comment{注释}
  \If  {$a=b$} \Comment{If的注释}
      \State 执行相关命令 \Comment{注释}
  \EndIf
  \If  {$c=d$}  
      \State 命令   \Comment{注释}
  \Else	
      \State  命令   \Comment{注释}
  \EndIf
\EndWhile
\Repeat
    \State   $j=j+1$ \Comment{注释}
\Until   {$j >  10$}
\For    {$i=1$ \textbf{to} $10$}
      \State $i=i+1$
\EndFor
\State \Return $o_1$, $o_2$
\end{algorithmic}
\end{algorithm*}


\section{公式}\label{c1sec:equation}

略。
更多公式排版见参考文献\cite{latexlive}。


\section{参考文献}\label{c1sec:reference}

参考文献样式分别上标(如参考文献\upcite{BITthesis,latexlive})和非上标(如参考文献\cite{BITthesis,latexlive},参考文献\cite{BITthesis}和\cite{latexlive})等样式。


\section{其他}\label{c1sec:other}

更多信息请参考文献\cite{BITthesis}。

关于LaTeX的更多知识参考文献\cite{LaTeX2e介绍}。


\chapter{待补充的内容}\label{c2}
由于时间有限,目前只在本模板中给出了影响论文排版美观程度的图、表、算法等示例。
后面可加入基于该模板的详细示例,比如基于该模板的华东理工大学百度百科介绍(包括大段文字、图、表、引用等),给读者呈现更直观的效果。

欢迎各位同学贡献想法、示例和批评指正。




%% 参考文献内容(小4号宋体),使用 BibTeX,包含参考文献文件.bib
%\bibliography{reference/chap1,reference/chap2} %多个章节的参考文献
\bibliography{reference/references} %


%%%%%%%%%%%%%%%%%%%%%%%%%%%%%%
%% 后置部分
%%%%%%%%%%%%%%%%%%%%%%%%%%%%%%

%% 附录(章节编号重新计算,使用字母进行编号)
\appendix
\renewcommand\theequation{\Alph{chapter}--\arabic{equation}}  % 附录中编号形式是"A-1"的样子
\renewcommand\thefigure{\Alph{chapter}--\arabic{figure}}
\renewcommand\thetable{\Alph{chapter}--\arabic{table}}
% 附录
%\include{chapters/appendix1}

%(其后部分无编号)
\backmatter
% 致谢

\begin{thanks}

华东理工大学薛梦奇基于BIT-Thesis模板做了一定修改,使其适用于华东理工大学的博士学位论文格式。
本文在其基础上进一步修改了部分内容(见第\ref{c1:intro}章第一段说明)。
对制作BIT-Thesis模板的各位老师、同学和薛梦奇师兄表示衷心的感谢。

\end{thanks}

% 发表文章目录


\chapter*{攻读博士期间的主要学术成果及参与的科研项目}
\addcontentsline{toc}{chapter}{攻读博士期间的主要学术成果及参与的科研项目}


\noindent\textbf{学术论文:}
\begin{enumerate}
    \item 学术论文1
    \item 学术论文2
\end{enumerate}



\noindent\textbf{发明专利:}
\begin{enumerate}
    \item 发明专利1
\end{enumerate}


\noindent\textbf{软件著作:}
\begin{enumerate}
    \item 软件著作1
\end{enumerate}



\noindent\textbf{科研项目:}
\begin{enumerate}
    \item 项目1
    \item 项目2
\end{enumerate}






% \includepdf[pages=-]{chapters/卷内备考表.pdf}

\end{document}
